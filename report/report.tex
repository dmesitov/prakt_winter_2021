\documentclass[a4paper,12pt]{extarticle}
% !TEX encoding = IBM866

\usepackage[framed,numbered,autolinebreaks,useliterate]{mcode}
\usepackage[utf8]{inputenc}
\usepackage[T2A]{fontenc}
\usepackage{listings}
\usepackage[cp866]{inputenc}
\usepackage{amsthm,amsmath,amsfonts}
\usepackage[dvipsnames]{xcolor}
\usepackage{algorithm2e}
\usepackage{array}
\usepackage[final]{graphicx}
\usepackage{float}
\usepackage{graphicx}
\usepackage{moreverb}
\usepackage{multirow}
\usepackage[cp866]{inputenc}
\usepackage[russian]{babel}
\usepackage{amsthm,amsmath,amsfonts,amssymb}
\usepackage[final]{graphicx}
\usepackage{float}
\usepackage{geometry}
\usepackage{subcaption}
\usepackage{colortbl}
\geometry
{
	a4paper,
	total={210mm,297mm},
	left=20mm,
	right=20mm,
	top=25mm,
	bottom=20mm,
}

\usepackage{tikz}
\usepackage{pgfplots}
\usepackage[final]{showkeys}

\usepackage{caption}
\DeclareCaptionLabelSeparator{dot}{. }
\captionsetup{labelsep=dot}

\usepackage{hyperref}
\usepackage{lastpage}
\usepackage{fancyhdr}
\usepackage{python}
\usepackage{listings}
\lstloadlanguages{Python}
\usepackage{bm}



\begin{document}
	
	
		\begin{titlepage}
		
		\begin{center}
			\centerline{\Large\rm Министерство науки и высшего образования}
			\centerline{\Large\rm Федеральное государственное бюджетное образовательное}
			\centerline{\Large\rm учреждение высшего образования}
			\centerline{\Large\rm <<Московский государственный технический университет}
			\centerline{\Large\rm имени~Н.~Э.~Баумана}
			\centerline{\Large\rm (национальный исследовательский университет)>>}
			\centerline{\Large\rm (МГТУ~им.~Н.~Э.~Баумана)}
			\hrulefill
		\end{center}
		
		\begin{figure}[h!]
			\centering
%			\includegraphics[height=0.4\linewidth]{picture.0}
		\end{figure}
		
		\begin{center}
			\centerline{\Large\rm Факультет <<Фундаментальные науки>>}
			\centerline{\Large\rm Кафедра ФН1 <<Высшая математика>>}
		\end{center}
		
		\begin{center}
			\textsc{\textbf{\Huge Отчет}}\\
			\textsc{\textbf{\large по учебной практике за 5 семестр}}\\
		\end{center}
		
		\vspace{3em}
		
		{
			\large
			\hbox to 17cm {Преподаватель \hspace{45pt} \hrulefill \hspace{60pt} Кравченко~О.~В.}
			\vspace{-7pt}
			\hbox{{\small\it \hspace{178pt} подпись, инициалы}}
			\hbox{}
			\hbox to 17cm {Студент группы ФН1--51Б \hrulefill \hspace{60pt} Меситов~Д.~М.}
			\vspace{-7pt}
			\hbox{{\small\it \hspace{178pt} подпись, инициалы}}
		}
		
		
		\vspace{\fill}
		
		\begin{center}
			\large	Москва \\2020
		\end{center}
		
	\end{titlepage}

    \setcounter{page}{2}
    \tableofcontents
    \vspace{\baselineskip}
   
    
    \newpage
    \section{Задание}
     Решить сингулярную двуточечную краевую задачу методом конечных разностей
    $$
    \varepsilon y''(x) + p(x)y'(x) + q(x)y(x) + f(x) = 0,\quad x\in[0,\,1],
    $$
    со смешанными граничными условиями
    $$
    \begin{array}{rcl}
    -\alpha_1y'(0) + \alpha_2y(0)			&=&	\gamma_1, ~~\alpha_1 = 1, ~\alpha_2 = 0, ~\gamma_1 = 5\\[8pt]
    \phantom{-}\beta_1y'(1) + \beta_2y(1)	&=&	\gamma_2, ~~\beta_1 = 1, ~\beta_2 = 0, ~\gamma_2 = 8
    \end{array}
    $$
    при различных значениях параметра 
    $$
    \varepsilon=1,~0.1,~0.01,~0.001.
    $$
    и функциях
    $$
    p(x)=x^{2}+x,~~~\\[8pt]
    q(x)=x-x^{2},~~~\\[8pt]
    f(x)=x^{3}-\sqrt{x^{2}+x}
    $$
    
    \section{Описание метода}
    Поставленная краевая задача решается с помощью перехода от исходной
    задачи к новой, записанной в конечно-разностной форме. Тогда решение
    новой задачи будет являться приближенным решением исходной задачи. В
    силу того, что первая и вторая производные, входящие в уравнение и в
    краевые условия, будут заменены приближенными конечно-разностными
    формулами, решения с применением метода конечных разностей получается
    не в виде непрерывной функции $y(x)$, а виде таблицы ее значений в
    отдельных точках. Для этого разобьем отрезок $[0, 1]$ на $n$ частей 
    так, чтобы $x_0 = 0, x_n = 1, x_{i+1} = x_i + h, h = (b-a)/n$. Наша
    задача – найти значения функции $y(x)$ в точках $x_k, k=0...n$. Для
    того, чтобы перейти от исходной задачи к конечно-разностной, надо
    получить формулы для представления первой и второй производных в
    конечно-разностном виде. Они получаются, если применить разложение
    функции $y(x)$ в окрестности некоторой точки $x_k$ в ряд Тейлора,
    ограничиваясь вторыми производными:
    $$
    y(x_i + h) \approx y(x_i) + hy'(x_i) + \frac{h^2}{2}y''(x_i)
    $$
    $$
    y(x_i - h) \approx y(x_i) - hy'(x_i) + \frac{h^2}{2}y''(x_i)
    $$
    Складывая (вычитая) эти выражения, получим приближенные выражения для
    второй (первой) производных:
    $$
    y''(x_i) \approx \frac{y(x_i + h) - 2y(x_i) + y(x_i - h)}{h^2}
    $$
    $$
    y'(x_i) \approx \frac{y(x_i + h) - y(x_i - h)}{2h}
    $$
    Обозначим $y(x_i) = y_i,~ y(x_i +h) = y_{i+1},~ y(x_i -h) = y_{i-1}$, а также $p(x_i) = p_i, ~ q(x_i) = q_i, ~ f(x_i) = f_i$.\\
    В итоге, получим для узловых точек сетки:
    \begin{equation}
        \varepsilon \frac{y_{i+1} - 2y_i + y_{i-1}}{h^2} + p_i \frac{y_{i+1}-y_{i-1}}{2h} + q_i y_i = -f_i,  ~~ i = 1...n-1
    \end{equation}\\
    И, для краевых условий:
    \begin{equation}
        -\alpha_1\frac{y_{1}-y_{0}}{h} + \alpha_2 y_0 = \gamma_1
    \end{equation}
    \begin{equation}
        \beta_1\frac{y_{n}-y_{n-1}}{h} + \beta_2 y_n = \gamma_2
    \end{equation}\\
    Мы получили СЛАУ с неизвестными $y_0...y_n$, матрица которой имеет
    трёхдиагональный вид. Такую систему можено решить, например, методом
    прогонки.
    
    \section{Листинг кода}
    \subsection{Метод конечных разностей}
    \begin{lstlisting}
    clc;
    eps = 1;
    a1 = 1;
    a2 = 0;
    b1 = 1;
    b2 = 0;
    g1 = 5;
    g2 = 8;
    h = 0.05;
    
    p = @(x)(x * x + x);
    q = @(x)(x - x * x);
    f = @(x)(x * x * x - sqrt(x * x + x));
    
    x0 = 0;
    x1 = 1;
    
    n = round(abs(x1 - x0) / h);
    X = zeros(n + 1, 1);
    F = zeros(n + 1, 1);
    A = zeros(n + 1, n + 1);
    
    for i = 1:n+1
        X(i) = x0 + (i - 1) * h;
    end
    for i = 2:n
        A(i, i)   = ((-2*eps) / (h * h)) + q(X(i));
        A(i, i-1) = (eps / (h * h)) - p(X(i)) / (2 * h);
        A(i, i+1) = (eps / (h * h)) + p(X(i)) / (2 * h);
        F(i)      = -f(X(i));
    end
    
    F(1)   = g1;
    F(n+1) = g2;
    A(1, 1) = (a1 / h) + a2;
    A(1, 2) = -a1 / h;
    A(n + 1, n + 1) = (b1 / h) + b2;
    A(n + 1, n)     = -b1 / h;
    
    Y = shuttleNew(A, F);
    plot(X, Y);
    \end{lstlisting}
    
    \subsection{Метод прогонки}
    \begin{lstlisting}
    function X = shuttleNew(A, B)

    n = size(A,1);
    a=diag(A);
    b=diag(A,1);
    c=diag(A,-1);
    P=zeros(n,1);
    Q=zeros(n+1,1);
    X=zeros(n,1);
    
    P(2)= b(1)/(-a(1));
    Q(2)= B(1)/(a(1));
    
    for i = 2:(n-1)
        P(i+1)=b(i)/(-a(i)-c(i-1)*P(i));
        Q(i+1)=(-B(i)+c(i-1)*Q(i))/(-a(i)-c(i-1)*P(i));
    end
    
    Q(n+1)=(-B(n)+c(n-1)*Q(n))/(-a(n)-c(n-1)*P(n));
    X(n)=Q(n+1);
    
    for i = (n-1):-1:1
        X(i) = P(i+1)*X(i+1)+ Q(i+1);
    end
    
    end
    \end{lstlisting}
    
    \section{Результаты работы программы}
   
    \begin{figure}[H]
    \begin{subfigure}{0.5\textwidth}
    \includegraphics[width=\linewidth]{pics/e1.jpg} 
    \caption{$\varepsilon=1, ~~ h = 0.05$}
    \label{fig:subim1}
    \end{subfigure}
    \begin{subfigure}{0.5\textwidth}
    \includegraphics[width=\linewidth]{pics/e01.jpg} 
    \caption{$\varepsilon=0.1, ~~ h = 0.05$}
    \label{fig:subim1}
    \end{subfigure}
    \end{figure}
    
    \begin{figure}[H]
    \ContinuedFloat
    \begin{subfigure}{0.5\textwidth}
    \includegraphics[width=\linewidth]{pics/e001.jpg} 
    \caption{$\varepsilon=0.01, ~~ h = 0.05$}
    \label{fig:subim1}
    \end{subfigure}
    \begin{subfigure}{0.5\textwidth}
    \includegraphics[width=\linewidth]{pics/e0001.jpg} 
    \caption{$\varepsilon=0.001, ~~ h = 0.05$}
    \label{fig:subim1}
    \end{subfigure}
    
    \end{figure}
    
    Заметим, однако, что с уменьшением параметра $\varepsilon$ матрица
    СЛАУ становится всё хуже обусловленной, а это в свою очередь приводит
    к большим погрешностям в ответе. Для сингулярных задач используются
    особые модифицированные алгоритмы, в том числе и модифицированные
    методы конечных разностей.\\
    С помощью комманды \mcode{dsolve} пакета MATLAB можно найти точное
    решение для задачи при значении параметра $\varepsilon = 1$.
    Изобразим его на одном графике с решением, полученным методом
    конечных разностей.
    
    \begin{center}
    \begin{figure}[H]
    \begin{subfigure}{0.75\textwidth}
    \includegraphics[width=\linewidth]{pics/compare.jpg} 
    \caption{Синим -- точное решение, оранжевым -- МКЭ с $\varepsilon=1, ~~ h = 0.001 $}
    \label{fig}
    \end{subfigure}
    \end{figure}
    \end{center}
    
    Для других значений $\varepsilon$ точное решение найти не удаётся.
    Для них только проанализируем, как меняется решение в зависимости от шага сетки: $h = \frac{1}{2}^k, ~~ k = 5...10$. (Чем линии менее прозрачные, тем меньше шаг сетки)
     
    \begin{figure}[H]
    \begin{subfigure}{0.5\textwidth}
    \includegraphics[width=\linewidth]{pics/net_thickening/e1.jpg} 
    \caption{$\varepsilon=1$}
    \label{fig}
    \end{subfigure}
    \begin{subfigure}{0.5\textwidth}
    \includegraphics[width=\linewidth]{pics/net_thickening/e01.jpg} 
    \caption{$\varepsilon=0.1$}
    \label{fig}
    \end{subfigure}
    \begin{subfigure}{0.5\textwidth}
    \includegraphics[width=\linewidth]{pics/net_thickening/e001.jpg} 
    \caption{$\varepsilon=0.01$}
    \label{fig}
    \end{subfigure}
    \begin{subfigure}{0.5\textwidth}
    \includegraphics[width=\linewidth]{pics/net_thickening/e0001.jpg} 
    \caption{$\varepsilon=0.001$}
    \label{fig}
    \end{subfigure}
    \end{figure}\\
    
    Можно заметить, что уменьшение шага также значительно влияет на полученный результат. Это тоже можно объяснить плохой обусловленностью полученной матрицы.
    
    

\end{document}
